\chapter{Information Resources}
\label{chap:InformationResources}

It is a fundamental human need to desire things.\footnote{Buddhists explain this desire as the source of all suffering. Alas, their noble solution to avoid desire to avoid suffering will not solve our problems in the Information Technology space.} At an early age, children learn to yell, scream, cry and babble to draw the attention of those who can satisfy their desires. Eventually, based on feedback, they learn how to refer to the objects of their desire: ``book'', ``ball'', ``cookie''. When ambiguity arises, they learn to clarify by pointing in the direction to signify which book, ball or cookie they want. Name. Point. Request.

It is to the immense credit of Tim Berners-Lee and those who helped him to design and build The Web, that this fundamental communication pattern was maintained. When you want something, you learn its name and you ask for it. The whole notion of disambiguation and resolvability are wrapped up in the design of \emph{Uniform Resource Locators} \citep{rfc1738}. The name of the thing is not only its identifier in a global context, it is also the handle by which you request it. Name. Point. Request.

Another trick that they pulled in their design was to give people the ability to come up with their own names. You did not need to seek the permission of a central authority to name something and share it. This flew in the face of institutionalized information managers who promised chaos if you allowed individuals to name their stuff. Entire industries were born to manage the centralized control of identifiers\footnote{Including hefty identifier registration fees!}. The benefit that they promised was guaranteed uniqueness.

Centralized control and permission-seeking can have their place in the stewardship of Information Technology, but they also impose their own burdens. It generally takes longer for a shared piece of information to be ``blessed''. Central authority also invariably serves to shape the information in the world view held by those empowered to make decisions. They will argue that standardization is a goal.

\section{Documents}

\section{Data}

\section{Services}

\section{Concepts}

\section{Identifiers}

\section{REST}
Architectural style
Stepping stone
Hypermedia
Content Negotiation
Sub-Application Level Consistency

\section{Scalability}

\section{Security}

\section{Versioning}